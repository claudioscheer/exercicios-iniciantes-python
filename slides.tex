\PassOptionsToPackage{unicode}{hyperref}
\documentclass[aspectratio=169]{beamer}

\usetheme{Madrid}
\usecolortheme{default}
\setbeamertemplate{navigation symbols}{}

\usepackage[T1]{fontenc}
\usepackage[utf8]{inputenc}
\usepackage{lmodern}
\usepackage[portuguese]{babel}
\usepackage{graphicx}
\usepackage{booktabs}
\usepackage{tabularx}
\usepackage{xcolor}
\usepackage{listings}
\usepackage{listingsutf8}
\usepackage{ulem}
% Habilita Unicode em metadados (bookmarks etc.)
\hypersetup{unicode=true}

% Cor mais visível para texto monoespaçado inline (\texttt{...})
\definecolor{ttcolor}{RGB}{0,51,153} % azul mais forte
\DeclareTextFontCommand{\textttblue}{\ttfamily\color{ttcolor}}
\DeclareTextFontCommand{\textttwhite}{\ttfamily\color{white}}
\let\texttt\textttblue

% Usar branco para \texttt em títulos de slides (frametitle e framesubtitle)
\makeatletter
\addtobeamertemplate{frametitle}{\begingroup\let\texttt\textttwhite}{\endgroup}
\addtobeamertemplate{framesubtitle}{\begingroup\let\texttt\textttwhite}{\endgroup}
\makeatother

\lstdefinestyle{py}{
  language=Python,
  basicstyle=\ttfamily\small,
  keywordstyle=\color{blue!70!black},
  commentstyle=\color{gray!70!black},
  stringstyle=\color{teal!60!black},
  showstringspaces=false,
  frame=single,
  rulecolor=\color{black!15},
  numbers=left,
  numberstyle=\tiny\color{gray!80!black},
  numbersep=8pt,
  tabsize=2,
  columns=fullflexible,
  keepspaces=true,
  upquote=true,
  breaklines=true,
  breakatwhitespace=true,
  xleftmargin=2em,
  framexleftmargin=2em,
  framexrightmargin=0em
}
\lstset{
  style=py,
  inputencoding=utf8,
  literate=
    {á}{{\'a}}1 {à}{{\`a}}1 {ã}{{\~a}}1 {â}{{\^a}}1
    {é}{{\'e}}1 {ê}{{\^e}}1
    {í}{{\'\i}}1
    {ó}{{\'o}}1 {ô}{{\^o}}1 {õ}{{\~o}}1
    {ú}{{\'u}}1 {ü}{{\"u}}1
    {ç}{{\c c}}1
    {Á}{{\'A}}1 {À}{{\`A}}1 {Ã}{{\~A}}1 {Â}{{\^A}}1
    {É}{{\'E}}1 {Ê}{{\^E}}1
    {Í}{{\'I}}1
    {Ó}{{\'O}}1 {Ô}{{\^O}}1 {Õ}{{\~O}}1
    {Ú}{{\'U}}1 {Ü}{{\"U}}1
    {Ç}{{\c C}}1
}

\title{Introdução à Programação com Python}
\subtitle{Sintaxe Básica}
\author{Claudio Scheer}
\date{\today}

\begin{document}

\begin{frame}
  \titlepage
\end{frame}

\begin{frame}{Por que Python?}
  \begin{itemize}
    \item Simples e legível: código fácil de entender e escrever.
    \item Sintaxe direta: menos símbolos, mais significado.
    \item Muito usado: do iniciante ao profissional.
    \item Comunidade grande e muitos exemplos.
  \end{itemize}
\end{frame}

% Removido o slide de Sintaxe Básica. Exemplos serão mostrados nos tópicos.

\begin{frame}[fragile]{Instruções e Variáveis}
  \begin{block}{Ideia}
    Uma instrução faz algo. Podemos guardar o resultado em uma variável.
  \end{block}
  \vspace{0.5em}
  \begin{lstlisting}
mensagem = "Olá"            # guarda um texto
numero = 10                 # guarda um número
nome = input("Seu nome: ")  # pede um texto
  \end{lstlisting}
  \vspace{0.5em}
  \textbf{Padrão:} \texttt{nome\_variavel = valor} ou \texttt{nome\_variavel = instrução()}
\end{frame}

\begin{frame}[fragile]{Funções: Definição}
  \begin{block}{O que é uma função?}
    Um ``pacotinho'' de passos para fazer uma tarefa.
  \end{block}
  \vspace{0.3em}
  \small Indentação (espaços no começo da linha) mostra o que faz parte da função.
\begin{lstlisting}
def faz_algo():
    nome = input("Seu nome: ")  # dentro da função (indentado)
    return nome                  # retorna o nome

def soma(a, b):
    resultado = a + b  # dentro da função
    return resultado
  \end{lstlisting}
\end{frame}

\begin{frame}[fragile]{Funções: Exemplos do Python}
  \begin{itemize}
    \item \texttt{print}: mostra um valor na tela.
    \item \texttt{input}: pede para digitar um valor.
  \end{itemize}
  \vspace{0.5em}
  \begin{lstlisting}
print("Oi!")                 # mostra um texto
texto = input("Digite algo: ")  # lê um texto
print("Você digitou:", texto)
  \end{lstlisting}
\end{frame}

\begin{frame}[fragile]{Tipos de Dados (exemplos simples)}
\begin{lstlisting}
idade = 5                 # número inteiro
altura = 1.70             # número decimal
nome = "Ana"              # texto (string)
ligado = True             # booleano (True ou False)
numeros = [1, 2, 3]       # lista
pessoa = {"nome": "Ana", "idade": 5}  # dicionário
  \end{lstlisting}
\end{frame}

\begin{frame}[fragile]{Contas (Operadores Aritméticos)}
  \begin{columns}[T,totalwidth=\textwidth]
    \column{0.47\textwidth}
    {\small
    \begin{tabularx}{\columnwidth}{@{}l l X@{}}
      \toprule
      \textbf{Operador} & \textbf{Descrição} & \textbf{Exemplo} \\
      \midrule
      \texttt{+}  & Somar               & \texttt{x + y} \\
      \texttt{-}  & Subtrair            & \texttt{x - y} \\
      \texttt{*}  & Multiplicar         & \texttt{x * y} \\
      \texttt{/}  & Dividir             & \texttt{x / y} \\
      \texttt{\%} & Resto da divisão    & \texttt{x \% y} \\
      \texttt{**} & Exponenciação       & \texttt{x ** y} \\
      \bottomrule
    \end{tabularx}}

    \column{0.47\textwidth}
\begin{lstlisting}[numbers=left,numberstyle=\tiny,numbersep=6pt,xleftmargin=0pt,framexleftmargin=0pt,framexrightmargin=0pt,basicstyle=\ttfamily\footnotesize]
a = 7
b = 3
soma = a + b        # 10
subtracao = a - b   # 4
produto = a * b     # 21
divisao = a / b     # 2.333...
resto = a % b       # 1
potencia = a ** b   # 343
\end{lstlisting}
  \end{columns}
\end{frame}

\begin{frame}[fragile]{Comparações (Operadores Lógicos)}
  \begin{tabularx}{\textwidth}{@{}l l X@{}}
    \toprule
    \textbf{Operador} & \textbf{Descrição} & \textbf{Exemplo} \\
    \midrule
    \texttt{>}   & maior que         & \texttt{x > y} \\
    \texttt{<}   & menor que         & \texttt{x < y} \\
    \texttt{==}  & igual a           & \texttt{x == y} \\
    \texttt{!=}  & diferente de      & \texttt{x != y} \\
    \texttt{>=}  & maior ou igual    & \texttt{x >= y} \\
    \texttt{<=}  & menor ou igual    & \texttt{x <= y} \\
    \bottomrule
  \end{tabularx}
  \vspace{0.5em}
\begin{lstlisting}
x = 10
y = 20
eh_igual = (x == y)      # False
eh_maior = (x > y)       # False
eh_menor_ou_igual = (x <= y)  # True
  \end{lstlisting}
\end{frame}

\begin{frame}[fragile]{Controle de Fluxo: \texttt{if}/\texttt{else}}
  \begin{block}{Ideia}
    Escolhe o que fazer com base em uma condição.
  \end{block}
\begin{lstlisting}
idade = 18
if idade >= 18:
    mensagem = "maior de idade"
else:
    mensagem = "menor de idade"
print(mensagem)
  \end{lstlisting}
\end{frame}

\begin{frame}{Somar dois números (passos)}
  \begin{enumerate}
    \item Pedir ao usuário um número entre 1 e 20.
    \item Pedir ao usuário um segundo número entre 1 e 20.
    \item Somar os dois números.
    \item Mostrar o resultado.
  \end{enumerate}
\end{frame}

\begin{frame}[fragile]{Somar dois números (exemplo)}
\begin{lstlisting}
primeiro = int(input("Digite um número entre 1 e 20: "))
segundo = int(input("Digite outro número entre 1 e 20: "))

total = primeiro + segundo
print("A soma é:", total)
  \end{lstlisting}
\vspace{0.3em}
{\footnotesize Resposta: \href{https://github.com/claudioscheer/exercicios-iniciantes-python/blob/master/exercicios/soma-inteiros.py}{@exercicios/soma-inteiros.py}}
\end{frame}

\begin{frame}{Atividade (Placar do Jogo)}
  \begin{block}{Desafio}
    O placar foi \textbf{3 x 2} para o time A, contra o time B.
  \end{block}
  \begin{itemize}
    \item O jogo empatou?
    \item Quem ganhou?
    \item Resposta: \href{https://github.com/claudioscheer/exercicios-iniciantes-python/blob/master/exercicios/verificar-empate.py}{@exercicios/verificar-empate.py}
  \end{itemize}
  \vspace{0.5em}
  \textit{Dica:} compare os placares usando os operadores de comparação.
\end{frame}

\begin{frame}{Atividade: Operações Básicas}
  \begin{itemize}
    \item Descrição: ler dois números e mostrar soma, subtração, multiplicação e divisão.
    \item Objetivo: praticar operadores aritméticos e exibir resultados.
    \item Resposta: \href{https://github.com/claudioscheer/exercicios-iniciantes-python/blob/master/exercicios/operacoes-matematicas.py}{@exercicios/operacoes-matematicas.py}
  \end{itemize}
\end{frame}

\begin{frame}{Atividade: Calculadora de Idade}
  \begin{itemize}
    \item Descrição: pedir ano de nascimento e calcular idade aproximada.
    \item Objetivo: ler entrada, converter e calcular.
    \item Resposta: \href{https://github.com/claudioscheer/exercicios-iniciantes-python/blob/master/exercicios/calculadora-idade.py}{@exercicios/calculadora-idade.py}
  \end{itemize}
\end{frame}

\begin{frame}{Atividade: Bhaskara}
  \begin{itemize}
    \item Descrição: ler \texttt{a}, \texttt{b}, \texttt{c} e calcular as raízes da equação do 2º grau.
    \item Objetivo: usar potência, raiz quadrada e condições para o delta.
    \item Resposta: \href{https://github.com/claudioscheer/exercicios-iniciantes-python/blob/master/exercicios/bhaskara.py}{@exercicios/bhaskara.py}
  \end{itemize}
\end{frame}

\begin{frame}{Atividade: Tabuada}
  \begin{itemize}
    \item Descrição: mostrar a tabuada de um número.
    \item Objetivo: praticar repetição e multiplicação.
    \item Resposta: \href{https://github.com/claudioscheer/exercicios-iniciantes-python/blob/master/exercicios/tabuada.py}{@exercicios/tabuada.py}
  \end{itemize}
\end{frame}

\begin{frame}{Atividade: Validar Caracteres}
  \begin{itemize}
    \item Descrição: pedir um texto e dizer se tem pelo menos N caracteres.
    \item Objetivo: trabalhar com strings e \texttt{len}.
    \item Resposta: \href{https://github.com/claudioscheer/exercicios-iniciantes-python/blob/master/exercicios/validar-caracteres.py}{@exercicios/validar-caracteres.py}
  \end{itemize}
\end{frame}

\begin{frame}{Atividade: Somar Números Pares}
  \begin{itemize}
    \item Descrição: somar todos os números pares em um intervalo (ex.: 0 a 100).
    \item Objetivo: praticar laços, condição de par e acumulação.
    \item Resposta: \href{https://github.com/claudioscheer/exercicios-iniciantes-python/blob/master/exercicios/somar-pares.py}{@exercicios/somar-pares.py}
  \end{itemize}
\end{frame}

\begin{frame}{Atividade: Verificar Empate}
  \begin{itemize}
    \item Descrição: ler placar de A e B; dizer se empatou e quem ganhou.
    \item Objetivo: comparar valores e imprimir mensagens.
    \item Resposta: \href{https://github.com/claudioscheer/exercicios-iniciantes-python/blob/master/exercicios/verificar-empate.py}{@exercicios/verificar-empate.py}
  \end{itemize}
\end{frame}

\begin{frame}{Atividade: Soma de Dois Números}
  \begin{itemize}
    \item Descrição: pedir dois inteiros e mostrar a soma.
    \item Objetivo: ler, converter e somar.
    \item Resposta: \href{https://github.com/claudioscheer/exercicios-iniciantes-python/blob/master/exercicios/soma-inteiros.py}{@exercicios/soma-inteiros.py}
  \end{itemize}
\end{frame}

\begin{frame}{Dicas Finais}
  \begin{itemize}
    \item Pratique: escreva e execute pequenos programas.
    \item Leia mensagens de erro com calma.
    \item Indentação correta é essencial.
    \item Comece simples e evolua aos poucos.
  \end{itemize}
\end{frame}

\begin{frame}
  \centering
  \Large Obrigado! \\
  \vspace{0.5em}
  \normalsize Perguntas?
\end{frame}

\end{document}